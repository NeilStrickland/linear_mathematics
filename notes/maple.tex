\documentclass{amsart}
\usepackage{fullpage}
\usepackage{hyperref}
\usepackage{verbatim}
\title{Introduction to Maple}

\begin{document}

\maketitle

Maple is a powerful software system for performing many different
symbolic and numerical calculations in mathematics.  This document will
give a brief introduction to using Maple for calculations in linear algebra.
There is a much more complete course on Maple at 
\begin{center}
 \url{https://neilstrickland.github.io/maths_with_maple/}
\end{center}

\section*{Using Maple}

When you first start Maple, it will offer you a choice between
Document Mode and Worksheet Mode.  I recommend Worksheet Mode.  When
Maple has started you can enter \texttt{2+2} and press
\texttt{ENTER}; Maple will print $4$ as an answer.  Now enter
\begin{verbatim}
 A := <<5|6|7>,<4|3|2>>
 B := <<8|9>,<9|8>>
\end{verbatim}
Maple will display this as 
\[ A := 
   \begin{pmatrix}
    5 & 6 & 7 \\
    4 & 3 & 2 
   \end{pmatrix}
\]
\[ B := 
    \begin{pmatrix}
     9 & 5 \\ 5 & 9
    \end{pmatrix}
\]
(The colon before the equals sign is essential.)  You can now enter
\verb+B.A+ to calculate $BA$, or \verb+B^2+ to calculate $B^2$, or
\verb+B^(-1)+ to calculate the inverse of $B$.  If you enter
\verb+A.B+ then Maple will print an error message explaining that $AB$
is undefined, because $A$ and $B$ do not have compatible shapes.

The $3\times 3$ identity matrix (for example) can be entered as
\verb|IdentityMatrix(3)|.  To save typing, it is best to enter 
\begin{verbatim}
I3 := IdentityMatrix(3)
\end{verbatim}
and then you can just type \verb|I3| after that.  You might think of
using the symbol \verb|I| instead of \verb|I3|, but Maple will not
allow that, because it uses \verb|I| for $\sqrt{-1}$.

To do more complicated things, you need some additional packages that
are not loaded by default.  I recommend that as soon as you start
Maple you enter 
\begin{verbatim}
with(LinearAlgebra): 
with(Student[LinearAlgebra]):
\end{verbatim}
This will load up everything that you need.  (Here and elsewhere in
Maple, you need to use capital letters exactly as specified; it will
not work to enter \verb+With(linearalgebra)+.)

In particular, after you have done the above you can enter
\begin{verbatim}
Transpose(A)
Determinant(B)
\end{verbatim}
to calculate $A^T$ and $\det(B)$.  You can extract rows and columns
from a matrix using syntax like
\begin{verbatim}
Row(A,2)
Column(B,1)
\end{verbatim}
You can refer to the bottom right hand entry of $A$ as \verb+A[2,3]+.

In many places, it is convenient to give names to the columns of a
matrix.  For example, we might take $u_1$, $u_2$ and $u_3$ to be the
columns of the matrix $A$ above.  To do this in Maple you can type
\begin{verbatim}
u[1] := Column(A,1)
u[2] := Column(A,1)
u[3] := Column(A,1)
\end{verbatim}
or
\begin{verbatim}
 for i from 1 to 3 do u[i] := Column(A,i) od
\end{verbatim}
(Here \verb+od+ is \verb+do+ backwards, and marks the end of the
\verb+do+ statement.  You can type \verb+end do+ instead of \verb+od+
if you prefer.)

To find the RREF of $A$, you can enter 
\begin{verbatim}
 ReducedRowEchelonForm(A)
\end{verbatim}
That is quite a lot of typing, so I suggest that after loading the
\verb+LinearAlgebra+ and \verb+Student[LinearAlgebra]+ packages you
immediately enter
\begin{verbatim}
 RREF := RowReducedEchelonForm
 GJET := GaussJordanEliminationTutor
\end{verbatim}
You can then type \verb+RREF(A)+ to find the RREF of $A$ immediately,
or \verb+GJET+to open a new window that leads you through the process
step by step.

If you are going to use the RREF of $A$ in further calculations, then
you probably want to give it a name, like this:
\begin{verbatim}
 A1 := RREF(A)
\end{verbatim}
Again, the colon before the equals sign is necessary.

To find and factor the characteristic polynomial of $B$ you can enter 
\begin{verbatim}
p := CharacteristicPolynomial(B,t)
factor(p)
\end{verbatim}
This works as written for square matrices like $B$ that have an even
number of rows.  However, Maple defines the characteristic polynomial
of $M$ to be $\det(M-tI)$, whereas the notes use $\det(tI-M)$.  (Both
of these conventions are common in the literature.)  Thus, for square
matrices of odd size you need to multiply Maple's characteristic
polynomial by $-1$ to get the characteristic polynomial as defined in
the notes.  

To find the eigenvalues and eigenvectors of $B$, enter
\verb|Eigenvectors(B)|.  The result consists of a vector and a matrix,
like this:
\[ \begin{pmatrix} 4\\ 14\end{pmatrix}
   \begin{pmatrix} -1&1 \\ 1&1 \end{pmatrix}
\]
The eigenvalues are $4$ and $14$, which appear as the entries in the
vector above.  The corresponding eigenvectors are
$\begin{pmatrix}-1\\1\end{pmatrix}$ and
$\begin{pmatrix}1\\1\end{pmatrix}$, which appear as the columns of the
matrix above.  In cases where the matrix has repeated eigenvalues the
result will need a bit more interpretation, but that will be discussed
later.

\end{document}
